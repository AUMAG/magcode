
\documentclass{article}
\usepackage{amsmath,array,booktabs,enumitem,upquote,shortvrb,underscore}
\usepackage[T1]{fontenc}
\usepackage{textcomp}
\usepackage[hyperref,backref]{biblatex}
\bibliography{journals,phd}
\usepackage{matlabweb}
\MakeShortVerb\|
\begin{document}
\title{Forces between magnets\\ and multipole arrays of magnets:\\ A Matlab implementation}
\author{Will Robertson}
\maketitle
\begin{abstract}
This is the user guide and documented implementation of a set of
Matlab functions for calculating the forces (and stiffnesses) between
cuboid permanent magnets and between multipole arrays of the same.

This document is still evolving. The documentation for the source code,
especially, is rather unclear/non-existent at present. The user guide,
however, should contain the bulk of the information needed to use this code.
\end{abstract}
\tableofcontents

\newpage

\section{User guide}

(See Section~\ref{meta} for installation instructions.)

\subsection{Forces between magnets}

The function |magnetforces| is used to calculate both forces and stiffnesses
between magnets. The syntax is as follows:

\begin{verbatim}
     forces = magnetforces(magnet_fixed, magnet_float, displ);
stiffnesses = magnetforces( ... , 'stiffness');
      [f s] = magnetforces( ... , 'force', 'stiffness');
\end{verbatim}

|magnetforces| takes three mandatory inputs to specify the position and magnetisation of the first and second magnets and the displacement between them.
Optional arguments appended indicate whether to calculate force or stiffness or both; the output arguments must match to reflect this choice.
The force\footnote{From now I will omit most mention of calculating stiffnesses; assume whenever I say `force' I mean `force \emph{and} stiffnesse'} is calculated as that imposed on the second magnet; for this reason, I often call the first magnet the `fixed' magnet and the second `floating'. If you wish to calculate the force on the first magnet instead, simply reverse the sign of the output.

\paragraph{Inputs and outputs}
The first two inputs are structures containing the following fields:
\begin{description}[noitemsep,font=\ttfamily]
\item[magnet.dim] A $(3\times1)$ vector of the side-lengths of the magnet.
\item[magnet.magn] The magnetisation magnitude of the magnet.
\item[magnet.magdir] A vector representing the direction of the magnetisation.
              This may be either a $(3\times1)$ vector in cartesian
              coordinates or a $(2\times1)$ vector in spherical coordidates.
\end{description}
In cartesian coordinates, the vector is interpreted as a unit vector; it is only used to calculate the direction of the magnetisation.
In other words, writing |[1;0;0]| is the same as |[2;0;0]|, and so on.
In spherical coordinates $(\theta,\phi)$, $\theta$ is the vertical projection of the angle around the $x$--$y$ plane ($\theta=0$ coincident with the $x$-axis), and $\phi$ is the angle from the $x$--$y$ plane towards the $z$-axis.
In other words, the following unit vectors are equivalent:
\begin{align*}
(1,0,0)_{\text{cartesian}} &\equiv (0,0)_{\text{spherical}}\\
(0,1,0)_{\text{cartesian}} &\equiv (90,0)_{\text{spherical}}\\
(0,0,1)_{\text{cartesian}} &\equiv (0,90)_{\text{spherical}}
\end{align*}
N.B.\ $\theta$ and $\phi$ must be input in degrees, not radians.
This seemingly odd decision was made in order to calculate quantities such as $cos(\pi/2)=0$ exactly rather than to machine precision.

The third mandatory input is |displ|, which is a matrix of displacement vectors between the two magnets. |displ| should be a $(3\times D)$ matrix,
where $D$ is the number of displacements over which to calculate the forces.
The size of |displ| dictates the size of the output force matrix; |forces| (etc.) will be also of size $(3\times D)$.

\paragraph{Example} Using |magnetforces| is rather simple. A magnet is set up
as a simple structure like
\begin{verbatim}
magnet_fixed = struct(...
  'dim'   , [0.02 0.012 0.006], ...
  'magn'  , 0.38, ...
  'magdir', [0 0 1] ...
);
\end{verbatim}
with something similar for |magnet_float|. The displacement matrix is then
built up as a list of $(3\times1)$ displacement vectors, such as
\begin{verbatim}
displ = [0; 0; 1]*linspace(0.01,0.03);
\end{verbatim}
And that's about it.
For a complete example, see `\path{examples/magnetforces_example.m}'.

\subsection{Forces between multipole arrays of magnets}

Because multipole arrays of magnets are more complex structures than single magnets, calculating the forces between them requires more setup as well.
The syntax for calculating forces between multipole arrays follows the same style as for single magnets:

\begin{verbatim}
     forces = multipoleforces(array_fixed, array_float, displ);
stiffnesses = multipoleforces( ... , 'stiffness');
      [f s] = multipoleforces( ... , 'force', 'stiffness');
\end{verbatim}

Because multipole arrays can be defined in various ways, there are several
overlapping methods for specifying the structures defining an array. Please escuse a certain amount of dryness in the information to follow; more inspiration for better documentation will come with feedback from those reading this document!

\paragraph{Linear arrays}
A minimal set of variables to define a linear multipole array are:
\begin{description}[noitemsep,font=\ttfamily]
\item[array.type] Either `|linear-x|', `|linear-y|', or `|linear-z|' to align the array with an axis.
\item[array.face] One of `|+x|', `|+y|', `|+z|', `|-x|', `|-y|', or `|-z|' to specify which direction the `strong' side of the array faces.
\item[array.msize] A $(3\times1)$ vector defining the size of each magnet in the array.
\item[array.Nmag] The number of magnets composing the array.
\item[array.magn] The magnetisation magnitude of each magnet.
\item[array.magdir_rotate] The amount of rotation, in degrees, between successive magnets.
\end{description}
Notes:
\begin{itemize}
\item The array must |face| in a direction orthogonal to its alignment.
\item `|up|' and `|down|' are defined as synonyms for facing `|+z|' and `|-z|', respectively, and `|linear|' for array type `|linear-x|'.
\item Singleton input to |msize| assumes a cube-shaped magnet.
\end{itemize}

The variables above are the minimum set required to specify a multipole array.
In addition, the following array variables may be used instead of or as well as to specify the information in a different way:
\begin{description}[noitemsep,font=\ttfamily]
\item[array.magdir_first] This is the angle of magnetisation in degrees around the direction of magnetisation rotation for the first magnet. It defaults to $\pm$90\textdegree\ depending on the facing direction of the array.
\item[array.length] The total length of the magnet array in the alignment direction of the array. If this variable is used then |width| and |height| (see below) must be as well.
\item[array.width] The dimension of the array orthogonal to the alignment and facing directions.
\item[array.height] The height of the array in the facing direction.
\item[array.wavelength] The wavelength of magnetisation. Must be an integer number of magnet lengths.
\item[array.Nwaves] The number of wavelengths of magnetisation in the array, which is probably always going to be an integer.
\item[array.Nmag_per_wave] The number of magnets per wavelength of magnetisation (e.g., |Nmag_per_wave| of four is equivalent to |magdir_rotate| of 90\textdegree).
\item[array.gap] Air-gap between successive magnet faces in the array. Defaults to zero.
\end{description}
Notes:
\begin{itemize}
\item |array.mlength|+|array.width|+|array.height| may be used as a synonymic replacement for |array.msize|.
\item When using |Nwaves|, an additional magnet is placed on the end for symmetry.
\item Setting |gap| does not affect |length| \emph{or} |mlength|! That is,
when |gap| is used, |length| refers to the total length of magnetic material placed end-to-end, not the total length of the array including the gaps.
\end{itemize}

\paragraph{Planar arrays}
Most of the information above follows for planar arrays, which can be thought
of as a superposition of two orthogonal linear arrays.
\begin{description}[noitemsep,font=\ttfamily]
\item[array.type] Either `|planar-xy|', `|planar-yz|', or `|planar-xz|' to align the array with a plane.
\item[array.width] This is now the `length' in the second spanning direction of the planar array. E.g., for the array `|planar-xy|', `length' refers to the $x$-direction and `width' refers to the $y$-direction. (And `height' is $z$.)
\item[array.mwidth] Ditto, etc.
\end{description}
All other variables for linear arrays hold analogously for planar arrays; if
desired, two-element input can be given to specify different properties in different directions.

\paragraph{Arbitrary arrays}
Until now we have assumed that magnet arrays are composed of magnets with identical sizes and regularly-varying magnetisation directions.
Some facilities are provided to generate more general/arbitrary--shaped arrays.
\begin{description}[noitemsep,font=\ttfamily]
\item[array.type] Should be `|generic|' but may be omitted.
\item[array.mcount] The number of magnets in each direction, say $(X,Y,Z)$.
\item[array.msize_array] An $(X,Y,Z,3)$-length matrix defining the magnet sizes for each magnet of the array.
\item[array.magdir_fn] An anonymous function that takes three input variables $(i,j,k)$ to calculate the magnetisation for the $(i,j,k)$-th magnet in the $(x,y,z)$-directions respectively.
\item[array.magn] At present this still must be singleton-valued. This will be amended at some stage to allow |magn_array| input to be analogous with |msize| and |msize_array|.
\end{description}
This approach for generating magnet arrays has been little-tested. Please inform me of associated problems if found.

\section{Meta-information}\label{meta}

\paragraph{Obtaining}
The latest version of this package may be obtained from the GitHub repository
\url{http://github.com/wspr/magcode} with the following command:
\begin{verbatim}
git clone git://github.com/wspr/magcode.git
\end{verbatim}

\paragraph{Installing}
It may be installed in Matlab simply by adding the `\texttt{matlab/}' subdirectory to the Matlab path; e.g., adding the following to your \texttt{startup.m} file: (if that's where you cloned the repository)
\begin{verbatim}
addpath ~/magcode/matlab
\end{verbatim}

\paragraph{Licensing}
This work may be freely modified and distributed under the terms and conditions of the Apache License v2.0.\footnote{\url{http://www.apache.org/licenses/LICENSE-2.0}}
This work is Copyright 2009--2010 by Will Robertson.

\paragraph{Contributing and feedback}
Please report problems and suggestions at the GitHub issue tracker.%
\footnote{\url{http://github.com/wspr/magnetocode/issues}}

The Matlab source code is written using Matlabweb.%
\footnote{\url{http://www.ctan.org/tex-archive/web/matlabweb/}}
After it is installed, use \verb|mtangle magnetforces| to extract the Matlab files \texttt{magnetforces.m} and \texttt{multipoleforces.m}, as well as extracting the test suite (such as it is, for now).
Running the Makefile with no targets (i.e., \verb|make|) will perform this step as well as compiling the documentation you are currently reading.

\section{Implementation}
\webfile{magnetforces}
\printbibliography
\end{document}
