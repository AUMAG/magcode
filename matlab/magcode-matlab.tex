
\documentclass{article}
\usepackage{amsmath,array,booktabs,enumitem,upquote,shortvrb,underscore}
\usepackage[utf8]{inputenc}
\usepackage[T1]{fontenc}
\usepackage{textcomp}
\usepackage{siunitx}

\usepackage{geometry}
\geometry{margin=3cm}

\usepackage{fancyvrb}
\fvset{fontsize=\small}

\usepackage{matlab-prettifier}
\usepackage{hyperref}
\hypersetup{colorlinks}
\usepackage[hyperref,backref,backend=bibtex]{biblatex}
\bibliography{journals,phd}

\newcounter{mfunction}
\newenvironment{mfunction}[1]{%
  \stepcounter{mfunction}%
  \ifcase\value{mfunction}\relax
    \ERROR
  \or
    \expandafter\subsection
  \or
    \expandafter\subsubsection
  \or
    \expandafter\paragraph
  \or
    \expandafter\subparagraph
  \fi
    {The \texttt{#1()} function}%
}{\addtocounter{mfunction}{-1}}


\RequirePackage{expl3,color,framed,ted}


\ExplSyntaxOn

\ior_new:N \pprint_doc_ior
\seq_new:N \l_pprint_line_seq
\tl_new:N \l_pprint_tl

\int_new:N \g_pprint_indent_int
\int_new:N \l_pprint_line_indent_int
\dim_new:N \l_pprint_dim

\tl_set:Nx \g_pprint_comment_tl { \cs_to_str:N \% }
\definecolor{commentstrcolor}{rgb}{0,0.7,0}
\definecolor{numberscolor}{gray}{0.6}

\def\linecommentstyle{%
  \sffamily
  \color{commentstrcolor}%
}

\cs_new:Nn \pprint_scan_setup:
 {
  \char_set_catcode_space:N \ 
%    \char_set_catcode_other:N \\
%    \char_set_catcode_space:N \ % space
%    \char_set_catcode_letter:N \#
%    \char_set_catcode_letter:N \$
%    \char_set_catcode_letter:N \^
%    \char_set_catcode_letter:N \_
%    \char_set_catcode_letter:N \& 
 }

%%%%

\cs_generate_variant:Nn \tl_if_eq:nnTF {x}
\cs_generate_variant:Nn \tl_if_eq:nnTF {o}
\cs_generate_variant:Nn \regex_extract_once:nnNTF {no}
\cs_generate_variant:Nn \tl_if_in:nnT {xx}
\cs_set_eq:NN \___tl_trim_spaces:n \tl_trim_spaces:n
\int_new:N \g_pprint_docline_int
\seq_new:N \l_tmp_seq
\seq_new:N \l_pprint_codelines_seq
\bool_new:N \l_pprint_start_bool

\tl_set:Nx \c_pprint_START_tl { \c_percent_str\c_space_tl \string \START }

\seq_put_right:Nn \l_file_search_path_seq {private/}

\cs_new:Npn \PrettyPrintFile #1
  {
    \group_begin:

      \bool_set_false:N \l_pprint_start_bool
      \ior_open:Nn \pprint_doc_ior {#1}
    
      \ior_str_map_inline:Nn \pprint_doc_ior
        {
          \int_gincr:N \g_pprint_docline_int
          
          \bool_if:NTF \l_pprint_start_bool
            {
              % avoid space trimming inside \seq_set_split:n following:
              % (MAJOR SIDE-EFFECTS OF COURSE)
              \cs_set_eq:NN \tl_trim_spaces:n \use:n
              \exp_args:NNo \seq_set_split:Nnn
                  \l_pprint_line_seq \g_pprint_comment_tl {##1}
              \cs_set_eq:NN \tl_trim_spaces:n \___tl_trim_spaces:n
            
              \seq_pop_left:NNT \l_pprint_line_seq \l_pprint_tl
                {
                  \tl_if_empty:NTF \l_pprint_tl
                    { \pprint_doc_line_nocode: }
                    { \pprint_doc_line_code:   }
                }
            }
            {
              \str_if_eq_x:nnT {\c_pprint_START_tl} {##1}
                { \bool_set_true:N \l_pprint_start_bool }
            }
        }
    \pprint_flush_codelines:
    \group_end:
  }

\cs_new:Nn \pprint_doc_line_nocode:
 {
  \seq_pop_left:NNTF \l_pprint_line_seq \l_pprint_tl
   % Comment on a line of its own:
   {
    \seq_put_right:No \l_pprint_codelines_seq \l_pprint_tl
   }
   % A completely empty line:
   {
    \pprint_flush_codelines:
    \medskip
   }
 }

\cs_new:Nn \pprint_matlab:n { \lstinline[style=Matlab-editor]!#1! }
\cs_new:Nn \pprint_matlab:N { \exp_after:wN \pprint_matlab:n \exp_after:wN {#1} }
\cs_new:Nn \pprint_doc_line_code:
 {
  \pprint_flush_codelines:

  \noindent
  \makebox[0em][r]{\tiny\color{numberscolor}\int_use:N \g_pprint_docline_int\quad}
%  \texttt{\l_pprint_tl}
  \tl_replace_all:Nnn \l_pprint_tl {~} {\ }
  \pprint_matlab:N \l_pprint_tl
  
  \seq_pop_left:NNT \l_pprint_line_seq \l_pprint_tl
   {
    \makebox
     {%
      \linecommentstyle
      \pprint_typeset_startcomment:\space
      \pprint_rescan_comment:n { \l_pprint_tl }
     }
   }
  \par
 }

\cs_new:Nn \pprint_rescan_comment:n
 {
  \exp_args:Nno \tl_rescan:nn
   {
    \pprint_scan_setup:
   }
   { #1 }
 }

% a bold percent sign to start a comment after a line of code:
\cs_new:Nn \pprint_typeset_startcomment:
 {
  \group_begin:
    \small
    \bfseries
    \g_pprint_comment_tl
  \group_end:
 }

\cs_new:Nn \pprint_flush_codelines:
 {
  \seq_if_empty:NF \l_pprint_codelines_seq
   {
    \medskip
    \seq_map_inline:Nn \l_pprint_codelines_seq
     {
      \tl_if_blank:nTF {##1} {\par}
       {
        \pprint_rescan_comment:n {##1}
        \space
       }
     }
    \seq_clear:N \l_pprint_codelines_seq
    \smallskip
    \par
   }
 }
\ExplSyntaxOff


\begin{document}
\MakeShortVerb\|
\title{Forces between magnets and multipole arrays of magnets:\\ A Matlab implementation}
\author{Will Robertson}
\maketitle
\vfil
\begin{abstract}
This is the user guide and documented implementation of a set of
Matlab functions for calculating the forces (and stiffnesses) between
cuboid permanent magnets and between multipole arrays of the same.

This document is still evolving. The documentation for the source code,
especially, is rather unclear/non-existent at present. The user guide,
however, should contain the bulk of the information needed to use this code.
\end{abstract}

\newpage
\tableofcontents

\newpage

\part{User guide}

(\emph{See Section~\ref{meta} for installation instructions.})

\section{Defining magnets and coils}

\begin{Verbatim}
  magnet = magnetdefine('type',T,key1,val1,...)
\end{Verbatim}

\paragraph{\texttt{'type'}}
The possible options for \texttt{T} are: \texttt{'cuboid'}, \texttt{'cylinder'}, \texttt{'coil'}.
If \texttt{'type',T} is omitted it will be inferred by the number of elements used to specify the dimensions of the magnets/coils.

\subsection{Cuboid magnets}

For cuboid magnets, the following should be specified:
\begin{description}[noitemsep,font=\ttfamily]
\item['dim'] A $(3\times1)$ vector of the side-lengths of the magnet.
\item['grade'] The `grade' of the magnet as a string such as \texttt{'N42'}.
\item['magdir'] A vector representing the direction of the magnetisation; either a $(3\times1)$ vector in cartesian
                coordinates or a string such as \texttt{'+x'}.
\end{description}
In cartesian coordinates, the |'magdir'| vector is interpreted as a unit vector; it is only used to calculate the direction of the magnetisation.
In other words, writing |[1;0;0]| is the same as |[2;0;0]|, and so on.

Instead of specifying a magnet grade, you may explicitly input the remanence magnetisation
of the magnet direction with
\begin{description}[noitemsep,font=\ttfamily]
\item['magn'] The remanence magnetisation of the magnet in Tesla.
\end{description}
Note that when not specified, the |magn| value $B_r$ is calculated from the magnet grade $N$ using $B_r=2\sqrt{N/100}$.

If you are calculating the torque on the second magnet, then it is assumed that the centre of rotation is at the centroid of the second magnet.
If this is not the case, the centre of rotation of the second magnet can be specified with
\begin{description}[noitemsep,font=\ttfamily]
\item['lever'] A $(3\times1)$ vector of the centre of rotation
(or $(3\times D)$ if necessary; see $D$ below).
\end{description}

\subsection{Cylindrical and ring magnets/coils}

If the dimension of the magnet (\texttt{'dim'}) only has two elements,
or the \texttt{'type'} is \texttt{'cylinder'}, the forces are calculated between two cylindrical or ring magnets.

Support for ring magnets is preliminary.

While coaxial and `eccentric' geometries can be calculated, the latter is around 50 times slower; you may want to benchmark your solutions to ensure speed is acceptable. (In the not-too-near-field, you can sometimes approximate a cylindrical magnet by a cuboid magnet with equal depth and equal face area.)
\begin{description}[noitemsep,font=\ttfamily]
\item['radius'] For cylindrical magnets, a $(1\times1)$ element specifying the magnet radius. For ring magnets, a $(2\times1)$ vector containing the inner and outer radius, resp.
\item['dim'] A $(2\times1)$ or $(3\times1)$ vector containing, respectively, the magnet radius and length (for cylinders) or magnet radii and length (for rings).
\item['dir'] Alignment direction of the cylindrical magnets;
`\textsf{x}' or `\textsf{y}' or `\textsf{z}' (default).
E.g., for an alignment direction of `\textsf{z}', the faces of the cylinder will be oriented in the $x$--$y$ plane.
\item['grade'] The `grade' of the magnet as a string such as \texttt{'N42'}.
\item['magdir'] A vector representing the direction of the magnetisation; either a $(3\times1)$ vector in cartesian coordinates or a string such as \texttt{'+x'}.
\end{description}
A `thin' magnetic coil can be modelled in the same way as a magnet, above; instead of
specifying a magnetisation, however, use the following:
\begin{description}[noitemsep,font=\ttfamily]
\item['turns'] A scalar representing the number of axial turns of the coil.
\item['current'] Scalar coil current flowing CCW-from-top.
\end{description}

\subsection{Coil---unfinished!}

A `thick' magnetic coil contains multiple windings in the radial direction and requires further specification.
The complete list of variables to describe a thick coil, which requires \texttt{'type'} to be `\textsf{coil}' are
\begin{description}[noitemsep,font=\ttfamily]
\item['dim'] A $(3\times1)$ vector containing, respectively, the inner coil radius, the outer coil radius, and the coil length.
\item['turns'] A $(2\times1)$ containing, resp., the number of radial turns and the number of axial turns of the coil.
\item['current'] Scalar coil current flowing CCW-from-top.
\end{description}
Again, only coaxial displacements and forces can be investigated at this stage.



\section{Forces}

\subsection{Forces between magnets}

The function |magnetforces| is used to calculate both forces and stiffnesses
between magnets. The syntax is as follows:

\begin{verbatim}
     forces = magnetforces(magnet_fixed, magnet_float, displ);
        ... = magnetforces( ... , 'force');
        ... = magnetforces( ... , 'stiffness');
        ... = magnetforces( ... , 'torque');
\end{verbatim}

|magnetforces| takes three mandatory inputs to specify `fixed' and `floating' magnets and the displacement between them.
Optional arguments appended indicate whether to calculate force and/or torque and/or stiffness respectively.
The force\footnote{From now I will omit most mention of calculating torques and stiffnesses; assume whenever I say `force' I mean `force \emph{and/or} stiffness \emph{and/or} torque'} is calculated as that imposed on the second magnet; for this reason, I often call the first magnet the `fixed' magnet and the second `floating'.

\paragraph{Outputs}
You must match up the output arguments according to the requested calculations.
For example, when only calculating torque, the syntax is
\begin{verbatim}
  T = magnetforces(magnet_fixed, magnet_float, displ,'torque');
\end{verbatim}
Similarly, when calculating all three of force/stiffness/torque, write
\begin{verbatim}
  [F, S, T] = magnetforces(magnet_fixed, magnet_float, displ,...
	         'force','stiffness','torque');
\end{verbatim}
The ordering of `force', `stiffness', `torque' affects the order of the output
arguments.
As shown in the original example, if no calculation type is requested then
the forces only are calculated.

\paragraph{Displacement inputs}

The third mandatory input is |displ|, which is a matrix of displacement vectors between the two magnets. |displ| should be a $(3\times D)$ matrix,
where $D$ is the number of displacements over which to calculate the forces.
The size of |displ| dictates the size of the output force matrix; |forces| (etc.) will be also of size $(3\times D)$.

\paragraph{Example} Using |magnetforces| is rather simple. A magnet is set up
as a simple structure like
\begin{verbatim}
magnet_fixed = magnetdefine(...
  'dim'   , [0.02 0.012 0.006], ...
  'magn'  , 0.38, ...
  'magdir', [0 0 1] ...
);
\end{verbatim}
with something similar for |magnet_float|. The displacement matrix is then
built up as a list of $(3\times1)$ displacement vectors, such as
\begin{verbatim}
displ = [0; 0; 1]*linspace(0.01,0.03);
\end{verbatim}
And that's about it.
For a complete example, see `\path{examples/magnetforces_example.m}'.


\subsection{Forces between multipole arrays of magnets}

Because multipole arrays of magnets are more complex structures than single magnets, calculating the forces between them requires more setup as well.
The syntax for calculating forces between multipole arrays follows the same style as for single magnets:

\begin{verbatim}
     forces = multipoleforces(array_fixed, array_float, displ);
stiffnesses = multipoleforces( ... , 'stiffness');
      [f s] = multipoleforces( ... , 'force', 'stiffness');
\end{verbatim}

Because multipole arrays can be defined in various ways, there are several
overlapping methods for specifying the structures defining an array. Please escuse a certain amount of dryness in the information to follow; more inspiration for better documentation will come with feedback from those reading this document!

\paragraph{Linear Halbach arrays}
A minimal set of variables to define a linear multipole array are:
\begin{description}[noitemsep,font=\ttfamily]
\item[array.type] Use `|linear|' to specify an array of this type.
\item[array.align] One of `|x|', `|y|', or `|z|' to specify an alignment axis along which successive magnets are placed.
\item[array.face] One of `|+x|', `|+y|', `|+z|', `|-x|', `|-y|', or `|-z|' to specify which direction the `strong' side of the array faces.
\item[array.msize] A $(3\times1)$ vector defining the size of each magnet in the array.
\item[array.Nmag] The number of magnets composing the array.
\item[array.magn] The magnetisation magnitude of each magnet.
\item[array.magdir_rotate] The amount of rotation, in degrees, between successive magnets.
\end{description}
Notes:
\begin{itemize}
\item The array must |face| in a direction orthogonal to its alignment.
\item `|up|' and `|down|' are defined as synonyms for facing `|+z|' and `|-z|', respectively, and `|linear|' for array type `|linear-x|'.
\item Singleton input to |msize| assumes a cube-shaped magnet.
\end{itemize}

The variables above are the minimum set required to specify a multipole array.
In addition, the following array variables may be used instead of or as well as to specify the information in a different way:
\begin{description}[noitemsep,font=\ttfamily]
\item[array.magdir_first] This is the angle of magnetisation in degrees around the direction of magnetisation rotation for the first magnet. It defaults to $\pm$90\textdegree\ depending on the facing direction of the array.
\item[array.length] The total length of the magnet array in the alignment direction of the array. If this variable is used then |width| and |height| (see below) must be as well.
\item[array.width] The dimension of the array orthogonal to the alignment and facing directions.
\item[array.height] The height of the array in the facing direction.
\item[array.wavelength] The wavelength of magnetisation. Must be an integer number of magnet lengths.
\item[array.Nwaves] The number of wavelengths of magnetisation in the array, which is probably always going to be an integer.
\item[array.Nmag_per_wave] The number of magnets per wavelength of magnetisation (e.g., |Nmag_per_wave| of four is equivalent to |magdir_rotate| of 90\textdegree).
\item[array.gap] Air-gap between successive magnet faces in the array. Defaults to zero.
\end{description}
Notes:
\begin{itemize}
\item |array.mlength|+|array.width|+|array.height| may be used as a synonymic replacement for |array.msize|.
\item When using |Nwaves|, an additional magnet is placed on the end for symmetry.
\item Setting |gap| does not affect |length| \emph{or} |mlength|! That is,
when |gap| is used, |length| refers to the total length of magnetic material placed end-to-end, not the total length of the array including the gaps.
\end{itemize}

\paragraph{Planar Halbach arrays}
Most of the information above follows for planar arrays, which can be thought
of as a superposition of two orthogonal linear arrays.
\begin{description}[noitemsep,font=\ttfamily]
\item[array.type] Use `|planar|' to specify an array of this type.
\item[array.align] One of `|xy|' (default), `|yz|', or `|xz|' for a plane with which to align the array.
\item[array.width] This is now the `length' in the second spanning direction of the planar array. E.g., for the array `|planar-xy|', `length' refers to the $x$-direction and `width' refers to the $y$-direction. (And `height' is $z$.)
\item[array.mwidth] Ditto for the width of each magnet in the array.
\end{description}
All other variables for linear Halbach arrays hold analogously for planar Halbach arrays; if desired, two-element input can be given to specify different properties in different directions.

\paragraph{Planar quasi-Halbach arrays}
This magnetisation pattern is simpler than the planar Halbach array described above.
\begin{description}[noitemsep,font=\ttfamily]
\item[array.type] Use `|quasi-halbach|' to specify an array of this type.
\item[array.Nwaves] There are always four magnets per wavelength for the quasi-Halbach array. Two elements to specify the number of wavelengths in each direction, or just one if the same in both.
\item[array.Nmag] Instead of |Nwaves|, in case you want a non-integer number of wavelengths (but that would be weird).
\end{description}

\paragraph{Patchwork planar array}
\begin{description}[noitemsep,font=\ttfamily]
\item[array.type] Use `|patchwork|' to specify an array of this type.
\item[array.Nmag] There isn't really a `wavelength of magnetisation' for this one; or rather, there is but it's trivial. So just define the number of magnets per side, instead. (Two-element for different sizes of one-element for an equal number of magnets in both directions.)
\end{description}

\paragraph{Arbitrary arrays}
Until now we have assumed that magnet arrays are composed of magnets with identical sizes and regularly-varying magnetisation directions.
Some facilities are provided to generate more general/arbitrary--shaped arrays.
\begin{description}[noitemsep,font=\ttfamily]
\item[array.type] Should be `|generic|' but may be omitted.
\item[array.mcount] The number of magnets in each direction, say $(X,Y,Z)$.
\item[array.msize_array] An $(X,Y,Z,3)$-length matrix defining the magnet sizes for each magnet of the array.
\item[array.magdir_fn] An anonymous function that takes three input variables $(i,j,k)$ to calculate the magnetisation for the $(i,j,k)$-th magnet in the $(x,y,z)$-directions respectively.
\item[array.magn] At present this still must be singleton-valued. This will be amended at some stage to allow |magn_array| input to be analogous with |msize| and |msize_array|.
\end{description}
This approach for generating magnet arrays has been little-tested. Please inform me of associated problems if found.


\section{Meta-information}\label{meta}

\paragraph{Obtaining}
The latest version of this package may be obtained from the GitHub repository
\url{http://github.com/wspr/magcode} with the following command:
\begin{verbatim}
git clone git://github.com/wspr/magcode.git
\end{verbatim}

\paragraph{Installing}
It may be installed in Matlab simply by adding the `\texttt{matlab/}' subdirectory to the Matlab path; e.g., adding the following to your \texttt{startup.m} file: (if that's where you cloned the repository)
\begin{verbatim}
addpath ~/magcode/matlab
\end{verbatim}

\paragraph{Licensing}
This work may be freely modified and distributed under the terms and conditions of the Apache License v2.0.\footnote{\url{http://www.apache.org/licenses/LICENSE-2.0}}
This work is Copyright 2009--2010 by Will Robertson.

This means, in essense, that you may freely modify and distribute this
code provided that you acknowledge your changes to the work and retain
my copyright. See the License text for the specific language governing
permissions and limitations under the License.

\paragraph{Contributing and feedback}
Please report problems and suggestions at the GitHub issue tracker.%
\footnote{\url{http://github.com/wspr/magcode/issues}}

\printbibliography

\clearpage
\part{Typeset code / implementation}

\section{Magnets setup}

\PrettyPrintFile{magnetdefine.m}
\PrettyPrintFile{magnetdraw.m}
\PrettyPrintFile{magnetforces.m}

\section{Magnet interactions}

The functions described in this section are translations of specific cases from the literature.
They have been written to be somewhat self-contained from the main code so they can be used directly for translation into other programming languages, or in applications where speed is important (such as for dynamic simulations).

\PrettyPrintFile{dipole_forcetorque.m}

\PrettyPrintFile{cuboid_force_z_z.m}
\PrettyPrintFile{cuboid_force_z_y.m}
\PrettyPrintFile{cuboid_force_z_x.m}
\PrettyPrintFile{cuboid_torque_z_z.m}
\PrettyPrintFile{cuboid_torque_z_y.m}

\PrettyPrintFile{cylinder_force_coaxial.m}
\PrettyPrintFile{cylinder_force_eccentric.m}

\section{Mathematical functions}

\PrettyPrintFile{ellipkepi.m}

\section{Magnet arrays}

\PrettyPrintFile{multipoleforces.m}

\end{document}

